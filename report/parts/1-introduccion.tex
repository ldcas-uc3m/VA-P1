\section{Introducción}

En la presente práctica se plantea la aplicación de los conocmientos adquiridos acerca de redes convolucionales y procesado de imágenes para el tratamiento de imágenes dermatológicas de lunares, a fin de distinguir y clasificar si estos se tratan de melanoma (tumor maligno), nevus (benigno) o keratosis seborreica (benigno).
Durante al realización de la práctica se han desarrollado transformaciones para aplicar sobre el conjunto de imágenes dadas a clasificar, con el fin de mejorar el entrenamiento de la red 
permitiendo así extraer más características diferentes a dicha red. Durante la práctica también se ha desarrollado la arquitectura de la red CNN partiendo de la base original, una vez 
implementada la red y las transformaciones sobre las imágenes, se han llevado acabo múltiples pruebas para detectar posibles mejoras y evaluar los resultados obtenidos.










