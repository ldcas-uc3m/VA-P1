\section{Conclusiones}
El desarrollo de esta práctica nos ha parecido de gran utilidad para poder poner en uso los conocimientos teóricos obtenidos acerca del funcionamiento de las CNN y su arquitectura, 
así como el funcionamiento del tratamiento de imágenes y como la apliación de diversas transformaciones tiene efecto en la precisión del modelo desarrollado.\\
Tras evaluar el modelo final, los resulatdos arrojados indicaban que el modelo se podría considerar un buen clasificador obteniendo un average de 0.734, con los siguientes parámetros detallados AUCs: mel 0.623333 sk 0.845459 avg 0.734396.\smallskip

Durante el desarrollo, hemos encontardo ciertas dicicultades principalmnete a la hora de identificar que parámetros dentro de la arquitectura de la red son los que más afectan asu desempeño, así como el tiempo de computo, ya que especialmente en el caso de la evaulación final del dataset óptimo con \texttt{maxSize=0}, el tiempo de ejcución resultó bastante elevado.